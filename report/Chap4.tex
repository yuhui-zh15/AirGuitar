\chapter{实验与分析}

    \section{目标}
    为了研究项目中\textbf{左手九宫格和弦选择}和\textbf{右手扫弦高度反馈}的准确率,我们设计了如下实验并进行了分析。

    \section{受试者}
        \paragraph{人数} 7人,均为清华大学计算机科学与技术系学生
        \paragraph{年龄} 19-20岁
        \paragraph{性别} 6男1女
        \paragraph{对吉他熟悉程度} 4位熟悉吉他,3位不熟悉吉他
        \paragraph{对人机交互熟悉程度} 5位选修HCI课程,2位不选修HCI课程
        \paragraph{特别说明} 其中3人为程序开发者

    \section{实验环境}
        \paragraph{硬件} 硬件统一使用Macbook Pro和Leap Motion
        \paragraph{软件} 用户界面统一都使用了扁平化主题
        \paragraph{环境} 均在寝室进行实验,受试者坐于椅子上,Leap Motion放置在桌子上,光线良好,温度适宜

    \section{实验任务}
        注:由于几乎所有可寻找的受试者都面临着实验日期截止的软件工程文档和其他课文档的任务,因此简化实验流程至10分钟内可以完成,理论应该测试更多的项目和更多的次数。
        \paragraph{左手和弦} 根据实验者要求,被试者选择要求的10个和弦,记录正确选中的次数,学习时间定为3分钟,调查用户满意度(1-5分)。
        \paragraph{右手扫弦} 被试者完成10次相反方向扫弦与5次连续同方向扫弦,分别记录正确扫弦的次数,学习时间定为3分钟,调查用户满意度(1-5分)。
        这两个测试是最基本的测试,实际上如果想要完整的弹奏一首曲子,必须左手和右手同步进行测试,实际由于Leap Motion准确率的问题,同步测试时准确率可能会下降较多,但本测试是基础,仍然具有重要意义。

    \section{过程}
        \paragraph{演示} 通过演示,教会实验者使用Air Guitar,了解Leap Motion原理。
        \paragraph{学习} 用户学习左手和弦选择3分钟。
        \paragraph{实验} 根据实验者要求,被试者选择要求的10个和弦,记录正确选中的次数。
        \paragraph{休息} 被试者休息1分钟。
        \paragraph{学习} 用户学习右手扫弦3分钟。
        \paragraph{实验} 被试者完成10次相反方向扫弦与5次连续同方向扫弦,分别记录正确扫弦的次数。


    \section{实验数据}

        \paragraph{学习时间} 3分钟
        \tablefourL{实验序号}{左手和弦正确选择次数}{左手和弦正确选择准确率}{左手和弦选择满意度}
        1 & 5 & 50\% & 3 \\
        2 & 6 & 60\% & 4 \\
        3 & 8 & 80\% & 4 \\
        4 & 7 & 70\% & 3 \\
        \tableend

        \tablesixL{实验序号}{右手不同方向扫弦正确次数}{右手不同方向扫弦正确率}{右手同方向扫弦正确次数}{右手同方向扫弦正确率}{右手和弦选择满意度}
        1 & 8 & 80\% & 3 & 60\% & 4 \\
        2 & 10 & 100\% & 5 & 100\% & 4 \\
        3 & 10 & 100\% & 4 & 80\% & 4 \\
        4 & 10 & 100\% & 2 & 40\% & 3 \\
        \tableend


        \paragraph{学习时间} 1小时以上(对比实验,受试者均为开发者)
        \tablefourL{实验序号}{左手和弦正确选择次数}{左手和弦正确选择准确率}{左手和弦选择满意度}
        1 & 9 & 90\% & 4 \\
        2 & 10 & 100\% & 5 \\
        3 & 9 & 90\% & 5 \\
        \tableend


        % 实验序号 右手不同方向扫弦正确次数 右手同方向扫弦正确次数 右手不同方向扫弦正确率 右手同方向扫弦正确率
        \tablesixL{实验序号}{右手不同方向扫弦正确次数}{右手不同方向扫弦正确率}{右手同方向扫弦正确次数}{右手同方向扫弦正确率}{右手和弦选择满意度}
        1 & 10 & 100\% & 4 & 80\% & 5 \\
        2 & 10 & 100\% & 5 & 100\% & 5 \\
        3 & 10 & 100\% & 5 & 100\% & 5 \\
        \tableend

    \section{数据分析}

        通过以上实验数据我们可以看出:

        \paragraph{学习时间对准确率的影响} 通过对比我们可以发现,开发者进行实验的准确率远远高于普通用户,这是因为开发者更加熟悉设备,同时也进行了大量的探索与尝试,可见学习时间对准确率有着很大的影响。但是,在完全不了解系统仅仅学习每个功能三分钟的时间下,用户在各个测试中准确率平均值也全部超过50\%,证明系统的易用性还是相对较好的。

        \paragraph{左手和弦} 普通用户准确率平均65\%,开发者准确率平均93\%,普通用户平均满意度3.5,开发者平均满意度4.7。在短短三分钟内,能取得这样的结果已经较好,但是还有待提高。

        \paragraph{右手扫弦} 普通用户不同方向扫弦准确率平均95\%,开发者准确率平均100\%,普通用户同方向扫弦准确率平均70\%,开发者准确率平均93\%,普通用户平均满意度3.8,开发者平均满意度5.0。可以看出不同方向扫弦用户已经可以熟练掌握,但是相同方向扫弦仍然是瓶颈,不过在短短三分钟内,能取得这样的结果已经较好,但是还有待提高。

        整体来看,用户反馈已经较好,目前准确率囿于学习时间的限制,通过更长的学习时间,准确率将会取得更大提升。

        完整而系统的测试还应包括同步测试(受试者同时左手选和弦右手扫弦)、乐曲测试(受试者根据吉他谱弹奏乐曲)、分解和弦测试(受试者独立弹奏每一根弦),但考虑到这些测试需要用户对吉他有一个较为全面的了解,时间成本过高,留给今后的工作。
