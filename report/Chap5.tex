\chapter{总结与展望}

    \section{实现了什么,有什么优点?}
        TODO

    \section{瓶颈以及解决方案}

        \subsection{识别范围}
        轻松的弹奏操作需要给用户足够大的动作幅度空间,然而Leap Motion并不能提供这么大的识别范围\
        若右手离开识别范围,则再次进入时会有不能忍受的较大延时。若减少右手作用幅度,扫弦非常流畅且容易控制,\
        达到预期目标。

        一个可能的解决方案是增加手到Leap Motion的高度,由于Leap Motion识别范围为锥体,因此\
        这样可以增加平面上的活动范围。但是手举太高容易使弹奏者疲劳,也不符合吉他弹奏的感觉。

        \subsection{双手干扰}
        开发过程中我们注意到,单独对左右手的调试都可以达到我们想要实现的目标,比如左手选一个和弦\
        序列,右手扫出想要的pattern等。但是一旦到了两者结合起来,不仅双手能活动的范围变小,\
        双手的坐标准确度也没有只有一只手的时候高。

        可以尝试用两个Leap Motion,由于一开始调研的时候也想到可能双手会比较挤的情形,\
        在写用户界面的时候,我们的数据交换是基于TCP的,这样的方式可以方便地应用到\
        多个Leap Motion的数据分工获取与交换,便于后续工作(如果有的话)。
