\chapter{问题背景}

    \section{吉他概述}
    吉他,属于弹拨乐器。通常有六条弦,形状与提琴相似。吉他在流行音乐、摇滚音乐、蓝调、民歌、佛朗明哥中,常被视为主要乐器。而在古典音乐的领域里,吉他常以独奏或二重奏的型式演出;在室内乐和管弦乐中,吉他亦扮演着相当程度的陪衬角色。因其风格多样、轻便易携带、价格低廉,在当今时代愈发流行。

    \image[6in]{a}{吉他}

    吉他的通常有两种弹奏方式:

    \begin{enumerate}
        \item{扫弦}:扫弦多用于伴奏,扫弦谱一般只标注和弦和扫弦节奏。可概述为左手和弦、右手扫弦。
        \item{指弹}:指弹可以用于伴奏,一般为独奏,指弹谱会标注和弦和品位。可概述为左手按品,右手拨弦。
    \end{enumerate}

    \image[6in]{c}{扫弦与指弹}

    扫弦一般是旋律轻快的地方,而指弹多是一些委婉,悲苦,或者温柔,舒缓的节奏,这是指在伴奏当中两者的区别。

    我们的项目着重于实现\textbf{吉他扫弦},这是因为扫弦是吉他乐器的特色之一,大部分乐器无法扫弦,并且扫弦可以弹唱绝大部分乐曲,对准确度要求也相对较低,娱乐性和用户体验更高。

    扫弦的方式是,左手和弦,右手扫弦。将和弦图旋转90度,便对应上了吉他六根弦的位置,左手按住和弦图标注的位置,右手上下扫动没有被禁用的弦(和弦图标注x的地方是扫弦不能扫到的弦),便会发出一段连续而优美的乐音。

     \image[6in]{d}{扫弦方式}

    \section{选题背景}

        \subsection{自然}
        相比平板电脑上有GUI的数字乐器,使用手势识别的系统最大程度地保留了弹奏着与乐器互动\
        的可玩性;同时由于使用了摄像头代替触控屏,使用户的体验更加轻松、自然,活动范围更大。\
        尤其使对于右手,扫弦时不会有触控屏带来的非常影响体验的阻力。

        \subsection{便捷}
        空气吉他相比于传统吉他,有数字乐器一样便于携带、不用调音的优点。

        \subsection{全民}
        尽管本应用使用的是Leap Motion,但事实上在任何有Depth Camera甚至仅RGB Camera\
        系统上,只要部署了手势识别系统,都可以实现这样的吉他。因此在未来大多数人都不需要购买\
        乐器就可体验虚拟乐器,进而帮助发现自己喜欢的乐器。

    \section{现有项目}

        \subsection{Virtual Guitar}
        现有的类似项目的一个典型是名为Virtual Guitar的项目。该项目不考虑左手的动作,\
        右手也只能点击实现指弹,不能扫弦,本质上只是套用了点击功能并在相应位置,放上吉他的弦。

        该项目有用户手位置的实时反馈来帮助用户操作。在我们对左手选取和弦的实现中借鉴了这个思路,\
        并且研究了是否提供反馈对操作准确度的影响。
