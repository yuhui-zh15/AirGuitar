\chapter{实现细节}

    \section{整体框架}

        采用MVP设计模式,后端提供处理Leap Motion帧后得到的吉他信息接口,前端界面使用这些接口获取和弦设置、\
        右手位置等信息。

        通过面向对象的实现方式,我们的代码可维护性较高。

    \section{后端}
        为了提高开发效率,我们选择了Leap Motion的Python SDK,运行在本地,处理帧并且提供吉他的数据;\

    \section{前端}
        前端用户界面(GUI)采用HTML+CSS+JavaScript搭建,可以较快地实现不错的界面和动画,提供了选择和弦的动画和弦振动的动画。\

        同时这样可以把前后端分离开来,即使后端不是基于Leap Motion的,\
        只要接口一致,我们的用户界面可以通用。

        另一个这样考虑原因是可扩展性,如果需要扩展到多个Leap Motion,\
        因为需要用到多台主机(python sdk不支持一台主机连多个LM),使用Web传递数据将非常方便。

        最终界面详见下一章。

    \section{前后端通信}
    使用了短轮询解决前端(GUI)与后端(Leap Motion)的通信问题,这是由于Leap Motion的异步机制与WebSocket的异步机制不能很优雅地兼容,\
    导致后端无法向前端主动推送数据来反应用户双手的变化。只能前端轮询数据,虽然效率有所降低,\
    但由于本身数据量小,使用过程中并没有卡顿。

    \section{音乐库}
    使用了FluidSynth软音源,包含各种MIDI中典型乐器的音色,使用mingus.midi.fluidsynth\
    提供的接口,来模拟一个音的产生,音高范围为两位的16进制码表示,即0\~127,其中60为中央C,
    如果预先选择了乐器,那么就能播放这个乐器弹这个音的声音,该乐库轻量易使用,最终音质很接近真实吉他。

    \section{连续同方向扫弦问题}
    通过手指高度信息加以限制,通过弦的振动加以反馈,用户通过体验学习,可以达到可接受效果。
 
