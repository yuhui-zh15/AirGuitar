\chapter{问题背景}

    \section{为什么要选这个题?}

        \subsection{自然}
        相比平板电脑上有GUI的数字乐器,使用手势识别的系统最大程度地保留了弹奏着与乐器互动\
        的可玩性;同时由于使用了摄像头代替触控屏,使用户的体验更加轻松、自然,活动范围更大。\
        尤其使对于右手,扫弦时不会有触控屏带来的非常影响体验的阻力。

        \subsection{便捷}
        空气吉他相比于传统吉他,有数字乐器一样便于携带、不用调音的优点。

        \subsection{全民}
        尽管本应用使用的是Leap Motion,但事实上在任何有Depth Camera甚至仅RGB Camera\
        系统上,只要部署了手势识别系统,都可以实现这样的吉他。因此在未来大多数人都不需要购买\
        乐器就可体验虚拟乐器,进而帮助发现自己喜欢的乐器。


    \section{什么是吉他?}
    一种食物。就是右手可以把六根弦一口吞下或者一根一根吃。 TODO

    \section{现有的项目如何?}

        \subsection{Virtual Guitar}
        现有的类似项目的一个典型是名为Virtual Guitar的项目。该项目不考虑左手的动作\
        右手也只能点击实现指弹,本质上只是套用了点击功能并在相应位置,放上吉他的弦。该项目\
        有用户手位置的实时反馈来帮助用户操作。在我们对左手选取和弦的实现中借鉴了这个思路,\
        并且研究了是否提供反馈对操作准确度的影响。
