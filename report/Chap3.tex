\chapter{项目展示}

    \section{用户界面}
    制作提供了\textbf{拟物化主题}和\textbf{扁平化主题}两套主题。

    \image[6in]{g}{拟物化主题}

    \image[6in]{h}{扁平化主题}

    两套主题制作都较为精美细致,保留了吉他本身的特点。

    \section{和弦选择界面}

    当用户左手进行和弦选择时,红色框将在九宫格上滑动,来实时地反馈弹奏者手指地位置,当弹奏者\
    找到所需和弦时,降低左手位置超过一阈值就会触发和弦的设置。

    从交互角度来看,可以认为这是通过反馈来减少误选;
    从吉他弹奏角度来看,也可以理解成需要把弦按下力度达到一定程度才能达到效果,而不是轻碰就可以,\
    类似实际弹奏的逻辑。

    \image[6in]{i}{和弦选择界面(拟物化主题)}

    \image[6in]{j}{和弦选择界面(扁平化主题)}

    \section{触弦振动界面}
    右手的操作我们也提供了反馈。可以看到界面上展示了吉他以及六根弦,当弹奏者右手触到相应位置时,\
    发出和弦对应弦的乐音,对应弦将给出合适的振动效果,使其看起来更加逼真。

    \image[6in]{k}{弦振动界面(拟物化主题)}

    \image[6in]{l}{弦振动界面(扁平化主题)}

    \section{后台运行}
    后端的Python脚本会持续运行处理并呈现数据,输出日志便于调试。

    \image[6in]{m}{后端输出}

