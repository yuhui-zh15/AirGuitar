\chapter{总结与展望}

    \section{项目总结}
    % 实现了什么,有什么优点?
    % TODO:maybe not professional, please modify if necessary
    % TODO: need some data
    本次实验实现了一个基于Leap Motion的吉他扫弦应用,其主要成果包括如下几个方面:

        \subsection{右手扫弦交互}
        在AirGuitar中,右手负责扫弦。

            \paragraph{实现方式} 我们通过模拟真实吉他扫弦的交互方式,在用户右手的惯用位置水平虚拟构造了6根弦,符合吉他弹奏者的使用习惯。
            当用户的右手在Leap Motion的指定高度之上扫过,对应位置处的弦便会接连被触发,从而发出一串动人的旋律。

            \paragraph{实现效果} 经过测试,我们发现,右手扫弦交互可以在绝大多数情况下精准识别用户扫过的弦。

            大部分用户可以在3分钟左右的训练时间内达到90\%左右的扫弦正确率,而在经过1小时左右的熟悉过程之后则可达到100\%。
            并且,用户仅需简单的学习过程,便可以掌握高度的控制方法,从而使得扫弦与否的误操作率达到正常弹奏可以忍受的级别。

        \subsection{左手选择和弦交互}
        在AirGuitar中,左手负责选择和弦。

            \paragraph{实现方式} 我们通过将真实吉他选择和弦的方式加以简化,将其接口抽象为简洁的九宫格形式,同时在其中置入最常见的和弦,兼顾了简洁性和可用性。
            当用户的左手水平移动到指定的坐标,并以手指点击时,对应位置处的和弦便会被选择,从而改变用户扫弦时发出的音调。% TODO:音调?响度?音色?

            \paragraph{实现效果} 经过测试,我们发现,在为界面加入反馈确认机制之后,用户可以在较短的时间内学会使用左手选择和弦。

            大部分用户可以在3分钟左右的训练时间内达到65\%左右的和弦选择准确率,而在经过1小时的熟悉过程之后,则可达到90\%-100\%。

        \subsection{用户界面}
        在AirGuitar中,用户界面为用户提供可视化的交互方式,并提供反馈机制、视觉线索等。

            \paragraph{实现方式} 用户界面使用HTML5+CSS+JS实现,使用浏览器渲染,轻量且可跨平台。

            其主界面分为2大部分:扫弦区域(对应右手)与和弦选择九宫格(对应左手)。

            \begin{enumerate}
                \itembf{反馈机制}:界面为右手扫弦和左手选择和弦均提供了反馈机制。当用户右手扫弦时,被扫过的弦会播放一段震动动画;
                当用户左手选择和弦时,被选和弦会被高亮为不同颜色。

                \itembf{视觉线索}:界面提供了完善的视觉线索,UI动态将用户左手在九宫格中的当前位置渲染高亮。
                同时,用户按下选择和弦时,UI会将对应位置高亮为不同颜色。
            \end{enumerate}

            \paragraph{实现效果} 经过测试,我们发现,简洁美观的界面极大地降低了用户的学习成本,并提升了使用体验。

            用户在反馈机制与视觉线索的引导下,也能显著地降低误操作率,从而提高演奏的流畅度。这一点笔者对比开发初期和后期的调试效率可谓深有体会。

    \section{瓶颈以及解决方案}

        \subsection{识别范围}

            \paragraph{瓶颈分析}
            轻松的弹奏操作需要给用户足够大的动作幅度空间,然而Leap Motion并不能提供这么大的识别范围\
            若右手离开识别范围,则再次进入时会有不能忍受的较大延时。若减少右手作用幅度,扫弦非常流畅且容易控制,\
            达到预期目标。

            \paragraph{解决方案}
            一个可能的解决方案是增加手到Leap Motion的高度,由于Leap Motion识别范围为锥体,因此\
            这样可以增加平面上的活动范围。但是手举太高容易使弹奏者疲劳,也不符合吉他弹奏的感觉。

        \subsection{双手干扰}
            \paragraph{瓶颈分析}
            开发过程中我们注意到,单独对左右手的调试都可以达到我们想要实现的目标,比如左手选一个和弦\
            序列,右手扫出想要的pattern等。但是一旦到了两者结合起来,不仅双手能活动的范围变小,\
            双手的坐标准确度也没有只有一只手的时候高。

            \paragraph{解决方案}
            可以尝试用两个Leap Motion,由于一开始调研的时候也想到可能双手会比较挤的情形,\
            在写用户界面的时候,我们的数据交换是基于TCP的,这样的方式可以方便地应用到\
            多个Leap Motion的数据分工获取与交换,便于后续工作(如果有的话)。
