\chapter{实现细节}

    \section{整体框架?}

        \subsection{MVP}
        采用MVP模式,后端提供处理LM帧后得到的吉他信息接口,前端界面使用这些接口获取和弦设置、\
        右手位置等信息。

        \subsection{Leap Motion摄像头数据的获取}
        为了提高开发效率,我们选择了Python sdk,运行在本地,处理帧并且提供吉他的数据;\

        \subsection{用户界面}
        用户界面写的是网页前端,可以较快地实现不错的界面。\

        同时,这样可以把前后端分离开来,就算后端不是基于Leap Motion的,\
        只要接口一致,我们的用户界面可以通用。

        另一个这样考虑原因是可扩展性,如果我们需要用到多个Leap Motion(虽然最后没有做),\
        因为需要用到多台主机(python sdk不支持一台主机连多个LM),使用web传递数据将非常方便。

        最终界面请参考Chapter 3。


    \section{音乐库?}
    使用了FluidSynth软音源,包含各种MIDI中典型乐器的音色,使用mingus.midi.fluidsynth\
    提供的接口使用。来模拟一个音的产生,音高范围为两位的16进制码表示,即0~127,其中60为中央C,
    如果预先选择了乐器,那么就能播放这个乐器弹这个音的声音。

    需要注意的是,不同于钢琴,由于吉他上不同位置有相同的音,并且音色有细微区别,但是用我们比较简陋\
    的MIDI不能区分这两种,好在对听感影响很小,最终还是很接近真实吉他。

    \section{前后端通信}
    使用了短轮询,这是由于Leap Motion的异步机制与WebSocket的异步机制不能很优雅地兼容,\
    导致后端无法向前端主动推送数据来反应用户双手的变化。只能前端轮询数据,虽然效率有所降低,\
    但使用过程中并没有发觉到卡顿。
